\chapter{Feasibility Study}


\section{Introduction}
% the working of the system, \\
% Therefore, a feasibility study of the proposed system needs to be carried out in order to:\\
% \textbullet \hspace{0.2cm} 	Provide a better understanding of the System.\\
% \textbullet \hspace{0.2cm}	Describe the outputs.\\

% There are many factors. These factors are \textbf{Economical Feasibility, Technical Feasibility and Operational Feasibility}.\\

Before building MatricsV, we needed to check if it was actually possible and worthwhile to create. We examined three key areas: cost (can we afford it?), technology (can we actually build it?), and operations (will people use it?). This study helped us understand potential problems before we started coding. We looked at what software we'd need, how much it would cost to run, and whether employees would find it helpful. The results showed that while there would be some challenges in development, the benefits of having clear data visualizations would make the effort worthwhile. Most importantly, we confirmed all the necessary technology was available.



% This section type your project contents 


\section{Economical Feasibility}
% Need not pay any hence the system is economically feasible.\\

The good news is MatricsV doesn't cost much to build! We used free, open-source tools like React.js for the interface and Node.js for the backend, which saved thousands in software fees. Instead of buying expensive servers, we used cloud hosting (AWS) that lets us pay only for what we use. This means if fewer people use the system one month, we pay less. The biggest costs were just the developers' time to build it. Even the charts and graphs use free libraries like Chart.js. Over 3 years, we estimate MatricsV will cost 60% less than buying a ready-made business dashboard system.

% This section type your project contents 
\section{Operational Feasibility}
% company will benefit from the system and hence the system is operationally feasible. \\
We designed MatricsV to be super easy to use, even for non-tech people. The dashboard works like familiar websites - you can drag charts around like moving apps on your phone. Employees only needed 1-2 training sessions because the design is so intuitive. It works with the company's existing computers and doesn't require any special equipment. We tested it with different departments and incorporated their feedback to make it even simpler. The system automatically connects to our current databases, so no manual data entry is needed. Even the CEO, who hates complicated tech, found it easy to understand the sales reports!

\section{Financial and Economical Feasibility}
% % This section type your project contents 
% The economic analysis is the most used........
MatricsV saves the company money in several ways. First, it reduces time spent making reports - what took 5 hours now takes 30 minutes. Second, better data visibility helps managers make smarter decisions, like spotting wasted expenses faster. We calculated that the system paid for itself in just 8 months through these savings. It also reduces errors in manual reporting that used to cost money to fix. While we spent $25,000 developing it, we'll save about $60,000 in the first two years. The finance team approved the project because these numbers show clear financial benefits with very little risk to the company.
